\documentclass{article}
\title{CSL105 : Discrete Mathematics\\
\large{Minor Examination}\\
\large{Indian Institute of Technology Ropar}\\
\small{Instructor: Dr. Sudarshan Iyengar}
}
\date{February 28, 2017}
\usepackage{amsmath}
\begin{document}
\maketitle
Total Duration : 2 hours \hfill Total Marks : 80 M\\

\underline{\large{Section I \hspace*{7cm} [5 Marks each]}}
\begin{enumerate} 
    \item Let $S=\{1,2,3,\dots\}$. Consider a relation $R=\{(a,b)|a+b\leq 10\}$. Prove or disprove that $R$ satisfies reflexive, symmetric, antisymmetric and transitive properties.\\
    
Solution: \\
\begin{enumerate}
\item $R$ cannot be reflexive since $\forall a \in S \ni' a \geq 6, (a,a) \notin R$
\item R is symmetric since $a+b = b+a$ and hence $\forall (a,b) \in R$, we can say that $(b,a) \in R$
\item R is not antisymmetric since $(8,2) \in R$ and $(2,8) \in R$ but $2 \neq 8$. 
\item R is not transitive since, $(3,2) \in R$ and $(2,8) \in R$ but $(3,8) \notin R$ since $3+8 \geq 10$. 
\end{enumerate}
\begin{center}
- - - X - - -
\end{center}

    \item In the matrix representation of a relation, how does one find if the relation is transitive or not? Prove that your method works. \\ 
Solution: \\
Let $R$ be transitive and $M$ denote the matrix representing $R$.\\
For $R$ to be transitive, we know that $M^2 \leq M$\\

Let $M^2_{xy}$ be the the $x^{th}$ row $y^{th}$ column entry of $M^2$.\\ 
If $M^2_{xy} =1$ then there must exist at least one $y \in A \ni' M_{xy} = M_{yz} = 1$. This happens only if $xRy$ and $yRz$ guaranteeing that R is transitive.   

\begin{center}
- - - X - - -
\end{center}
    
    \item Every sequence of $n^2 + 1$ distinct real numbers contains a subsequence of length $n + 1$ that is either strictly increasing or strictly decreasing.\\
    
    Solution: \\
    
Let $a_1, a_2, \dots a_{n^2+1}$ be the sequence of $n^2+1$ distinct real numbers. \\
For $1 \leq k \leq n^2+1$, let $x_k$ denote the maximum length of  a decreasing subsequence that ends with $a_k$ and $y_k$ denote the maximum length of  a increasing subsequence that ends with $a_k$.\\ 
	If there is no decreasing or increasing subsequence of length $n+1$, then $1 \leq x_k \leq n$ and $1 \leq y_k \leq n$ for all values of $k$.  Consequently, there are at most $n^2$ distinct ordered pairs $(x_k,y_k)$. However we have a sequence of $n^2 +1$ ordered pairs associated with each term $a_k$ in the sequence. So, by pigeonhole principle, we have two identical ordered pairs $(x_i,y_i) , (x_j,y_j)$, where $i\neq j$. Since terms in the sequence are distinct, we arrive at a contradiction - either $a_i < a_j$ then $x_i > x_j$ or  $a_i > a_j$ then $y_i < y_j$.  	

\begin{center}
- - - X - - -
\end{center}
    
    \item State well ordering principle. State and Prove Mathematical Induction.\\
    Solution : \\
    The well-ordering principle states that every non-empty subset of positive integers contains a least element. \\

The principle of Mathematical induction states that:\\
Let $S(n)$ denote an open mathematical statement that involves one or more occurrences of the variable $n$, which represents a positive integer:
\begin{enumerate}
\item If $S(n)$ is true; and
\item If whenever $S(k)$ is true then $S(k+1)$ is true;
\end{enumerate}
then $S(n)$ is true for all $n \in Z^+$ \\

The proof of the above is as stated:\\
Let $S(n)$ be such an open statement satisfying (a), (b), and let $F=\{ t \in Z^+|S(t)\ is \ false \}$. 
We wish to prove that $F= \phi$, so to obtain a contradiction we assume $F \neq \phi$. By well ordering principle $F$ has a least element $m$. With $m-1 \notin F$, we have $S(m-1) = true$. So by condition (b) we have that $S(m-1)+1) = S(m) = true$, contradicting that $m \in F$. 


 \begin{center}
- - - X - - -
\end{center}
    
    \item What is the condition for a function to be invertible? Explain with an example. \\
    
    Solution: \\ 
    For a function $f$ to be invertible, it must be both one-one and onto.\\
    
    Let $f:A \rightarrow B$ be not onto. Then $\exists\  y \in B \ni' y$ does not have a pre-image. That is, $\forall x \in A, f(x) \neq y$. Therefore $f^{-1}(y)$ does not exist. \\ 
    Similarly, if $f$ is not one-one, $\exists\  y \in B \ni' f(x_1)=f(x_2)=y$, for some $x_1,x_2 \in A$. Again in this case, $f^{-1}(y)$ does not exist.
   
   
   
\begin{center}
- - - X - - -
\end{center}
    
    \item Six boxes are coloured red, black, blue, yellow, orange and green. In how man ways can you
put 20 identical balls into these boxes such that no box is empty?\\
Solution: \\
This can be done in ${(20-6) + 6 - 1 }\choose 5$ ways. \\
This is the same as the problem of having combinations with repetitions. Since no box is to be empty we have in total $(20-6)$ balls left after placing 1 ball in each box. 
 \begin{center}
- - - X - - -
\end{center}

    \item Prove that if R is a reflexive relation on set S, then so is any superset of R inside S $\times$ S.\\
    Solution: \\
    
    Since $R$ is already reflexive, $\forall a \in S, (a,a) \in R$. Hence a superset of $R$ will continue to  have the existing elements. Hence $R$ continues to be reflexive. 
\begin{center}
- - - X - - -
\end{center}
    
    \item Let $G = (V, E)$ be a loop free undirected graph. Prove that if G contains no cycle of odd
length then G is bipartite.\\
Solution: 

Suppose $G$ has no odd cycles.\\
Choose any vertex $v \in G$.\\

Divide $G$ into two sets of vertices like this:\\
Let $A$ be the set of vertices such that the shortest path from each element of $A$ to $v$ is of odd length;\\
Let $B$ be the set of vertices such that the shortest path from each element of $B$ to $v$ is of even length. \\
WLOG, let $v \in B$ and $A\cap B=\phi$.\\

Suppose $a_1,a_2 \in A$ are adjacent. Then there would be a closed walk of odd length cycle $(v,\dots,a_1,a_2,\dots,v)$.\\
This contradicts our initial supposition that $G$ contains no odd cycles.\\
So no two vertices in $A$ can be adjacent.\\
By the same argument, neither can any two vertices in $B$ be adjacent.\\
Thus $A$ and $B$ satisfy the conditions for $G=(A \cup B,E)$ to be bipartite. 
\begin{center}
- - - X - - -
\end{center}

\end{enumerate}

\underline{\large{Section II \hspace*{7cm} [10 Marks each]}}
\begin{enumerate} 
 \item Prove by Induction that $1+\frac{1}{2}+\frac{1}{3}+\dots=\infty$.
 \newline
Solution: \\
We see that the series is sum of Harmonic numbers\\
Let $H_{j}$ be the series $1+\frac{1}{2}+\frac{1}{3}+\dots+ \frac{1}{j}$\\
Now we prove that :\\
$H_{2^n} \geq 1 + \frac{n}{2}$\\
BASIS STEP: $P(0)$ is true, because $H_{2^0} = H_1 = 1 ≥ 1 + \frac{0}{2}$ \\
INDUCTIVE STEP: The inductive hypothesis is the statement that $P(k)$ is true, that is,\\
$H_{2^k} \geq 1 + \frac{k}{2}$ \\
where $k$ is an arbitrary non-negative integer. We must show that if $P(k)$ is true, then $P(k + 1)$, which states that $H_{2^{k+1}} \geq 1 + \frac{k+1}{2}$, is also true.\\
So, assuming the inductive hypothesis, it follows that:
\begin{align*}
H_{2^{k+1}} &=  1+\frac{1}{2}+\frac{1}{3}+\dots + \frac{1}{2^k}+  \frac{1}{2^k +1} \dots \frac{1}{2^{k+1}}\\
&= H_{2^k} + \frac{1}{2^k +1} \dots \frac{1}{2^{k+1}}\\
&\geq \left( 1+ \frac{k}{2}\right) +  \frac{1}{2^k +1} \dots \frac{1}{2^{k+1}}\\
& \geq \left( 1+ \frac{k}{2}\right) + 2^k.\frac{1}{2^{k+1}}\\
& \geq \left( 1+ \frac{k}{2}\right) + \frac{1}{2}\\
& = 1 + \frac{k+1}{2}
\end{align*}
Since we see that $H_{2^n}$ is diverging, we can see that  $1+\frac{1}{2}+\frac{1}{3}+\dots=\infty$. 




\begin{center}
- - - X - - -
\end{center}
 
 \item Show that $$1.2.3+2.3.4+3.4.5+\dots +n(n+1)(n+2)= \frac{n(n+1)(n+2)(n+3)}{4}$$. \\
 Solution : \\
 
 Proof by induction : \\
 $S(1) = 1.2.3 = \frac{1.2.3.4}{4}$, according to the closed form given. Hence, we can conclude $S(1)$ is true. \\
 
 Induction Hypothesis: Let the given statement be true up to some $k \ni' 1.2.3 + 2.3.4 \dots k(k+1)(k+2) = \frac{k(k+1)(k+2)(k+3)}{4}$\\
 
 Now, we need to prove that  $S(k+1)$ is true.\\
 Consider $1.2.3 + 2.3.4 \dots k(k+1)(k+2) + (k+1)(k+2)(k+3)$
 \begin{align*}
 &= \frac{k(k+1)(k+2)(k+3)}{4} + (k+1)(k+2)(k+3)\\
 &= (k+1)(k+2)(k+3)\left(\frac{k}{4} +1\right)\\
 &=  \frac{(k+1)(k+2)(k+3)(k+4)}{4}
 \end{align*}
Hence Proved.\\ 
 
 
 \begin{center}
- - - X - - -
\end{center}
 
 \item You need to choose a password which is at least 6 characters and at most 8 characters in length with an added condition that each character is an uppercase letter or a digit. Also, your password must contain at least one digit. In how many ways can you choose your password?\\ 
 Solution: \\
 $(36^6 - 26^6) + (36^7 - 26^7) +(36^8 - 26^8) $
 
\begin{center}
- - - X - - -
\end{center}
 \item Enumerate all possible non-isomorphic graphs on 4 vertices. \\ 
 Solution : \\ 
 0 edges: 1 unique graph.\\
1 edge: 1 unique graph.\\ 
2 edges: 2 unique graphs: one where the two edges are incident and the other where they are not incident. \\ 
3 edges: 3 unique graphs. One is a 3 cycle with an isolated vertex, and the other two are trees: one has a vertex with degree 3 and the other has 2 vertices with degree 2.\\ 
4 edges: 2 unique graphs: a 4 cycle and one containing a 3 cycle.\\ 
5 edges: 1 unique graph.\\ 
6 edges: 1 unique graph.

\end{enumerate}
\end{document}