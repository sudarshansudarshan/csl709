\documentclass{article}
\title{CSL105 : Discrete Mathematics\\
\large{Minor Examination}\\
\large{Indian Institute of Technology Ropar}\\
\small{Instructor: Dr. Sudarshan Iyengar}
}
\date{March 2017}
\begin{document}
\maketitle
Total Duration : 2 hours \hfill Total Marks : 80 M\\

\underline{\large{Section I \hspace*{7cm} [5 Marks each]}}
\begin{enumerate} 
    \item Let $S=\{1,2,3,\dots\}$. Consider a relation $R=\{(a,b)/a+b\leq 10\}$. Prove or disprove that $R$ satisfies reflexive, symmetric, antisymmetric and transitive properties. 
    \item In the matrix representation of a relation, how does one find if the relation is transitive or not? Prove that your method works.  
    \item Every sequence of $n^2 + 1$ distinct real numbers contains a subsequence of length $n + 1$ that is either strictly increasing or strictly decreasing. 
    \item State well ordering principle. State and Prove Mathematical Induction.
    \item What is the condition for a function to be invertible? Explain with an example. 
    \item Six boxes are colored red, black, blue, yellow, orange and green. In how man ways can you
put 20 identical balls into these boxes such that no box is empty?
    \item If R is a reflexive relation on S, then so is any superset of R inside S × S.
    \item Let $G = (V, E)$ be a loop free undirected graph. Prove that if G contains no cycle of odd
length than G is bipartite.

\end{enumerate}
\vspace*{3.5cm}


\underline{\large{Section II \hspace*{7cm} [10 Marks each]}}
\begin{enumerate} 
 \item Prove by Induction that $1+\frac{1}{2}+\frac{1}{3}+\dots=\infty$.
 \item Show that $$1.2.3+2.3.4+3.4.5+\dots +n(n+1)(n+2)= \frac{n(n+1)(n+2)(n+3)}{4}$$. 
 \item You need to choose a password which is at least 6 characters and at most 8 characters in length with an added condition that each character is an uppercase letter or a digit. Also, your password must contain at least one digit. In how many ways can you choose your password? 
 \item Enumerate all possible non-isomorphic graphs on 4 vertices. 

\end{enumerate}
\end{document}
