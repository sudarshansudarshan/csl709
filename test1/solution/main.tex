\documentclass[a4paper]{article}

%% Language and font encodings
\usepackage[english]{babel}
\usepackage[utf8x]{inputenc}
\usepackage[T1]{fontenc}

%% Sets page size and margins
\usepackage[a4paper,top=3cm,bottom=2cm,left=3cm,right=3cm,marginparwidth=1.75cm]{geometry}

%% Useful packages
\usepackage{amsmath}
\usepackage{graphicx}
\usepackage[colorinlistoftodos]{todonotes}
\usepackage[colorlinks=true, allcolors=blue]{hyperref}

\title{Solutions to Test 1}
\author{}

\begin{document}
\maketitle



\begin{enumerate}
\item There  are $n$ people present in a room.  Prove  that among them  there  are two people who have the  same number  of acquaintances in the  room.  (1 mark) 

\textbf{Answer:}
Given $n$ people there are two possibilities. 
\begin{itemize}
\item Case 1: When there is a possibility of a person having no acquaintances, then the total possible number of acquaintances one can have are $0,1,2,3,.....,n-2$.
 \item Case 2: When each person knows at least one other person, then the possible number of acquaintances one can have are $1,2,3,.....,n-1$.
 \end{itemize}
It is observable that in both the cases there are $n-1$ possibilities as the number of acquaintances(pigeon holes) while there are $n$ people(pigeons).
So, by pigeonhole principle, there are at least two people who have the same numner of acquaintances.
  
\item In 2016, there  were 35,000 rank  holders  in some national level entrance exam, with same rank being shared by multiple  people.  Alice, Bob, Cathy, Dirac and Elisa cleared this exam.  In how many ways can these 5 students secure their  ranks?  (2 marks)

\textbf{Answer:}
As there is a possibility for more than one person to secure a rank, each of Alice, Bob, Cathy, Dirac and Elisa have the possibility of securing one rank out of 35000 possibilities. Hence, for 5 students, the number of ways they can get ranks = $35000 \times 35000 \times 35000 \times 35000 \times 35000$. The product rule applies here, since for each rank secured by a student, there are $35000$ possibilites for every other.
So, the answer is 
\begin{math} 
 (35000)^5
\end{math}
\item S(8, 5) =?.  (2 marks)

\textbf{Answer:}
The value is 1050. $1$ mark is alloted for the value and a mark for the formula.
\begin{itemize}
\item Use S(m,n)=S(m-1,n-1) + n S(m-1,n)
\item Use
\begin{math} 
 S(m,n)=(1/n!)\sum_{j=0}^{n} (-1)^{n-j}\frac{n!}{j!(n-j)!}j^{n}
\end{math}
\end{itemize}

\item Give  an  example  of a function  $f : A \rightarrow B$  and  $A_1$ , $A_2 \subset A$ for which
$f(A_1 \cap A_2)$ = $f(A_1) \cap f(A_2)$.  (2 marks)

\textbf{Answer:}
Any function that satisfies the following criteria fetches 2 marks.

\item State  and prove the generalized  pigeon hole principle.(2 mark)

\textbf{Answer:}
If $N$ objects are place in $k$ boxes, then there must be at least one box with at least
\begin{math}
\left \lceil{\frac{N}{k}}\right \rceil 
\end{math}
objects.

Let us say we have $N$ objects and $k$ boxes and no box have more than
\begin{math}
\left \lceil{\frac{N}{k}}\right \rceil -1 
\end{math}
objects.Then, the total no of objects would be at max

\begin{math}
k*( \left \lceil{\frac{N}{k}}\right \rceil -1 ) < k*(((N/k)+1)-1)=k*(N/k)=N
\end{math}

So, no of objects would be then less than $N$. So, a contradiction and hence proved. 

\item During  a month  with 30 days,  a baseball  team  plays at  least  one game a day,  but  no more  than  45 games.   Show that there  must  be a period  of some number  of consecutive days during which the team must play exactly
14 games.  (2 marks)

\textbf{Answer:}

Let, $a_i$ denotes no of games played on the $i^{th}$ day. The team plays the following number of games in that month - 

$a_1, a_2, a_3, \dots a30$.
Given that, 
\begin{math}
\sum_{j=1}^{30} a_i  \leq 45.
\end{math}

Let $s_i$ denote the total number of games played up to the $i^{th}$ day.

\begin{math}
s_i=\sum_{j=1}^{i} a_i 
\end{math}

We observed that at least 1 game is played in each day. So, each $s_i \leq s_i+1$. 

$s_1 < s_2 < s_3 < \dots s_30 \leq 45$.

Define a new sequence $t$ such that $t_i=s_i+14$.

$t_1 < t_2 < t_3 < \dots t_30 \leq 59$.

Since each value in $s_i$ is unique , so each value in $t_i$ is also unique. Howwever, we have a set of  60 numbers.

By pigeon hole principle, there exists at least 2days which have same number of games played.

\item An odd number of people stand  in a park  at  mutually distinct distances.
At the  same time  each person  throws  a stone  at  their  nearest  neighbor, hitting this  person.  Use mathematical induction to show that there  is at least  one survivor,  that is, at  least  one person  who is not  hit  by a stone. (4 marks)

\textbf{Answer:}
We see that for 3 people there always exists a pair of who hit each other and the third hits one of them. Thus the third is left without hit. This is because, the pair with the minimum distance will choose to hit each other.
Let $p(i)$ denote $i$ people standing in the ground and 1 left without being hit. Let $p(3)$ is the base case.
Let us assume that we have the given statement is true for all odd values up to $k$, where $k=2x+1$. 
Now we need to prove that $p(k+2)$ is also true.
\begin{itemize}
\item case 1: there exist at least one pair who hit each other.
In this case we remove the pairs and by inducting on the remaining people, 1 person is surely left out.
\item case 2: No pair exists who hits each other. Therefore we obtain a odd length cycle. Without loss of generality we can represent this cycle as $1 \rightarrow 2 \rightarrow 3 \dots n-1 \rightarrow n \rightarrow 1$, where $x \rightarrow y$ denotes that $x$ hits $y$. Let $d(x,y)=d(y,x)$ denote the distance between person $x$ and person $y$.
Given that $n \rightarrow 1$, that is $n$ hit 1, we can say that:\\
$d(1,n)=d(n,1) < d(2,n),d(3,n).................d(n-1,n)$\\
Since 1 hit 2 we can say that $d(n,1)>d(1,2)$.
 
Due to the cycle we can also say that:
$d(1,2) > d(2,3) > d(3,4), \dots d(n-1,n) > d(n,1) > d(1,2)$
Thus, a contradiction and it can not be odd length cycle.There must be one pair who hit each other hence proved.
\end{itemize}

\end{enumerate}

\end{document}
