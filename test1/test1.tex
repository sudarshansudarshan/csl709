\documentclass{article}
\title{First test CSL105}
\date{3rd February 2017}
\begin{document}
\maketitle

\begin{enumerate}
    \item There are $n$ people present in a room. Prove that among them there are two people who
        have the same number of acquaintances in the room. (1 mark)
    \item In 2016, there were 35,000 rank holders in some national level entrance exam, with same rank being shared by 
        multiple people. Alice, Bob, Cathy, Dirac and Elisa cleared this exam. In how many ways can these 5 students secure their ranks? (2 marks)
    \item $S(8,5)=?$. (2 marks)
    \item Give an example of a function $f:A\rightarrow B$ and $A_1, A_2\subseteq A$ for which $f(A_1\bigcap A_2)\neq f(A_1)\bigcap f(A_2)$. (2 marks)
    \item State and prove the generalized pigeon hole principle. (2 mark)
    \item During a month with 30 days, a baseball team plays at least one game a day, but no more than 45 games. Show that there 
        must be a period of some number of consecutive days during which the team must play exactly 14 games. (2 marks)
    \item An odd number of people stand in a park at mutually distinct distances. At the same time each person throws a stone at their nearest 
        neighbor, hitting this person. Use mathematical induction to show that there is at least one survivor, that is, at least one person who is not hit by a stone. (4 marks)

\end{enumerate}
\end{document}
