\documentclass[a4paper]{article}

%% Language and font encodings
\usepackage[english]{babel}
\usepackage[utf8x]{inputenc}
\usepackage[T1]{fontenc}
\usepackage{datetime}
\newdate{date}{06}{01}{2017}
\newcommand\tab[1][1cm]{\hspace*{#1}}


%% Sets page size and margins
\usepackage[a4paper,top=3cm,bottom=2cm,left=3cm,right=3cm,marginparwidth=1.75cm]{geometry}

%% Useful packages
\usepackage{amsmath}
\usepackage{graphicx}
\usepackage[colorinlistoftodos]{todonotes}
\usepackage[colorlinks=true, allcolors=blue]{hyperref}
\usepackage{enumitem}   



\title{Tutorial 1}
\date{\displaydate{date}}
\begin{document}
\maketitle
\section{Counting Problems}
\begin{enumerate}
\item What is the value of $k$ after the following code snippet is executed :
\begin{enumerate}[label=(\alph*)]
\item 
k=0\\
for $i_1$ = $n_1$\\
\tab for $i_2$ = $n_2$\\
\tab \tab for $i_3$ = $n_3$\\
\tab \tab \tab \vdots\\
\tab \tab \tab for $i_m$ = $n_m$\\
\tab \tab \tab \tab k=k+1

\item 
k=0\\
for $i_1$ = $n_1$\\
\tab k=k+1\\
\tab for $i_2$ = $n_2$\\
\tab \tab k=k+1\\
\tab \tab for $i_3$ = $n_3$\\
\tab \tab \tab k=k+1\\
\tab \tab \tab \vdots\\
\tab \tab \tab for $i_m$ = $n_m$\\
\tab \tab \tab \tab k=k+1\\ 
\end{enumerate}

\item In how many ways can 10 men and 10 women be seated in a row if:
\begin{enumerate}[label=(\alph*)]
\item Any person can sit next to any other.
\item Men and women occupy alternate seat.
\item Husband and wife sit together.
\end{enumerate}

\item A committee of 8 has to be chosen out of 16 men and 10 women. In how many ways can this be done if:
\begin{enumerate}[label=(\alph*)]
\item No restrictions.
\item The committee must include equal men and women.
\item The committee must include 7 women
\item The committee must include more women than men.
\item The committee must include at least 6 men.
\end{enumerate}

\item Compute the value of the following:
\begin{enumerate}[label=(\alph*)]
\item ${n \choose 0} + {n \choose 1} + {n \choose 2} + \dots + {n \choose n}$ 
\item ${n \choose 0} - {n \choose 1} + {n \choose 2} - \dots + (-1)^n {n \choose n}$
\end{enumerate}
\end{enumerate}

\section{Permutations with Repetitions}
\begin{enumerate}
\item Amit, Nihal, Shrikanth, Rohan, Neelam and Rashmi have a Giani's free ice-cream coupon each, in which they can avail the following flavors : Chocolate, Vanilla, Strawberry, Butterscotch. In how many ways can they buy ice-creams? Is it the same as the number of ways in which the vendor at Giani can sell them 6 ice-creams? Justify your answer.

\item There are three bins, each containing red, green, and blue balls respectively. How many arrangements of 5 balls can be made if each bin has unlimited supply of balls?

\item In how many ways can 5 people A, B, C, D, E be arranged on a circular table, such that:
\begin{enumerate}[label=(\alph*)]
\item A and B are always seated together.
\item C and D never sit together.
\end{enumerate}

\item What is the number of subsets of a set with $n$ elements?

\item What is the number of solutions to:
\begin{equation}
x_1 + x_2 + x_3 + x_4 = 7 , \textnormal{where} x_i \geq 0 \textnormal{and} \forall i 1 \leq i \leq 4
\end{equation}

\item Count the number of ways in which 3 men and 3 women can be seated in a round table such that no two men sit together.

\item What would the coefficient of:
\begin{enumerate}[label=(\alph*)]
\item $x^5y^2$ be in the expansion of $(x+y)^7$?
\item $a^5b^2$ be in the expansion of $(2a-3b)^7$?
\end{enumerate}
\end{enumerate}



\end{document}