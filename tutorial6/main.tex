\documentclass[a4paper]{article}

%% Language and font encodings
\usepackage[english]{babel}
\usepackage{color}
\usepackage[utf8x]{inputenc}
\usepackage[T1]{fontenc}
\usepackage{datetime}
\newcommand\tab[1][1cm]{\hspace*{#1}}
\definecolor{indigo}{RGB}{75,0,130}


%% Sets page size and margins
\usepackage[a4paper,top=3cm,bottom=2cm,left=3cm,right=3cm,marginparwidth=1.75cm]{geometry}

%% Useful packages
\usepackage{amsmath}
\usepackage{graphicx}
\usepackage[colorinlistoftodos]{todonotes}
\usepackage[colorlinks=true, allcolors=blue]{hyperref}
\usepackage{enumitem}   



\title{Tutorial 6 - Graphs and Hamiltonian Cycles}
\date{17 March 2017}
\begin{document}
\maketitle
We took a look at the following ideas from the text book :

\begin{enumerate}
    \item \textbf{Cut set} : For a given connected graph, a cut-set is a minimal disconnecting set of edges. We also took a look at a few examples.

	\item \textbf{Hamilton cycle} : If $G = (V, E)$ is a graph or multigraph with $|V| \geq 3$, we say that $G$ has a Hamilton cycle
if there is a cycle in $G$ that contains every vertex in V.    
    
    \item At Professor Alfred's science camp, 17 students have lunch together each day at a circular table. They are trying to get to know one another better, so they make an effort to sit next to
two different colleagues each afternoon. For how many afternoons can they do this? How can they arrange themselves on these occasions?

	\item Theorem : Let $G = (V, E)$ be a loop-free undirected graph with $|V| = n \geq 3$. If $deg(x) + deg(y) \geq n$
for all nonadjacent vertices $x,y \in V$, then $G$ contains a Hamilton cycle.


\end{enumerate}




\end{document}
