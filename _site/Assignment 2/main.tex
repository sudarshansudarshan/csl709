\documentclass[a4paper]{article}

%% Language and font encodings
\usepackage[english]{babel}
\usepackage[utf8x]{inputenc}
\usepackage[T1]{fontenc}
\usepackage{datetime}
\newdate{date}{17}{02}{2017}
\newcommand\tab[1][1cm]{\hspace*{#1}}
\usepackage{amsfonts}

%% Sets page size and margins
\usepackage[a4paper,top=3cm,bottom=2cm,left=3cm,right=3cm,marginparwidth=1.75cm]{geometry}

%% Useful packages
\usepackage{amsmath}
\usepackage{graphicx}
\usepackage[colorinlistoftodos]{todonotes}
\usepackage[colorlinks=true, allcolors=blue]{hyperref}
\usepackage{enumitem}   



\title{Assignment 2}
\author{Deadline \date{\displaydate{date}} }
\begin{document}
\maketitle


\begin{enumerate}

\item Suppose that $p_1,p_2,p_3$ are distinct primes and that $n,k \in \mathbb{Z}^+$ with $n=p_1^5\ p_2^3\ p_3^k$. Let $A$ be the set of positive integer divisors of $n$ and define a relation $\mathbb{R}$ on $A$ by $x\mathbb{R}y$ if $x$ exactly divides $y$. If there are 5880 ordered pairs in $\mathbb{R}$, determine $k$ and $|A|$. 

\item Let $A$ be a set with $|A|=n$, and let $\mathbb{R}$ be an equivalence relation on $A$ with $|\mathbb{R}| = r$. Why is $r-n$ always even?

\item A relation $\mathbb{R}$ on a set $A$ is called \textit{irreflexive} if for all $a \in A, (a,a) \notin \mathbb{R}$. Let $\mathbb{R}$ be a non-empty relation on $A$. Prove that if $\mathbb{R}$ satisfies two of the following properties - reflexive, symmetric, transitive, then it cannot satisfy the third.

\item Given a set $A$ with $n$ elements and a relation $\mathbb{R}$ on $A$, let $M$ denote the relation matrix for $\mathbb{R}$. Then, prove the following:
\begin{enumerate}
\item $\mathbb{R}$ is \textit{reflexive} iff $I_n \leq M$.
\item $\mathbb{R}$ is \textit{symmetric} iff $M=M^T$
\item $\mathbb{R}$ is transitive iff $M.M = M^2 \leq M$
\end{enumerate}

\item Prove that $M(\mathbb{R}) = \textbf{0}$ iff $\mathbb{R} = \phi$.

\item Prove that $M(\mathbb{R}) = \textbf{1}$ iff $\mathbb{R} = A \times A$.

\item Prove that $M(\mathbb{R})^n = [M(\mathbb{R}]^n$, for all $n \in \mathbb{Z}^+$. 

\item Let $f: A \rightarrow B$. If {$B_1,B_2 \dots B_n$} is a partition of $B$, prove that {$f^{-1}(B_i)| 1\leq i \leq n, f^{-1}(B_i) \neq \phi $} is a partition of $A$. 

\item Suppose that $\mathbb{R}$ and $S$ are reflexive relations on a set $A$. Prove or disprove each of these statements:
\begin{enumerate}
\item $\mathbb{R} \cup S$ is reflexive
\item $\mathbb{R} \cap S$ is reflexive
\item $\mathbb{R} - S$ is irreflexive
$\mathbb{R} \circ S$ is reflexive
\end{enumerate}

\item Suppose that the relation $\mathbb{R}$ is irreflexive, is $\mathbb{R}^2$ necessarily irreflexive? Give reasons.

\item Let $\mathbb{R}$be the relation o the set of all metro stations in Delhi, such that $(a,b) \in \mathbb{R}$ if it is possible to go from stop $a$ to stop $b$ without changing trains. What is $\mathbb{R}^n$, for a positive integer $n$?

\item Let $n$ be a positive integer and $S$ a set of strings. Suppose that 
$R_n$ is the relation on $S$ such
that $sR_nt$ if and only if $s = t$, or both $s$ and $t$ have at least $n$ characters and the first $n$ characters
of $s$ and $t$ are the same. That is, a string of fewer than $n$ characters is related only to itself; a string $s$ with at least $n$ characters is related to a string $t$ if and only if $t$ has at least $n$ characters
and $t$ begins with the $n$ characters at the start of $s$. For example, let $n = 3$ and let $S$ be the set
of all bit strings. Then $sR_3t$ either when $s = t$ or both $s$ and $t$ are bit strings of length 3 or more
that begin with the same three bits. For instance, $01R_301$ and $00111R_3 00101$, but $01\not R_3 010$
and $01011\not  R_301110$.
Show that for every set $S$ of strings and every positive integer $n$, $R_n$ is an equivalence
relation on $S$.



\item Let $R_3$ be the relation from previous question. What are the sets in the partition of the set of all bit strings
arising from the relation $R_3$ on the set of all bit strings? 


\item Each bead on a bracelet with three beads is either red,
white, or blue. Define the relation $\mathbb{R}$ between bracelets as: $(B_1,B2)$,
where $B_1$ and $B_2$ are bracelets, belongs to $\mathbb{R}$ if and only
if $B_2$ can be obtained from $B_1$ by rotating it or rotating it
and then reflecting it.
\begin{enumerate}
\item Show that R is an equivalence relation.
\item What are the equivalence classes of $\mathbb{R}$?
\end{enumerate}

\item How many equivalence relations are there over the set A = (a, b, c)?


\item Given the partition P = {{1, 2}, {3}, {4, 5}} of the set A = {1, 2, 3, 4, 5},
consider R the associated equivalence relation on A. Draw the digraph
associated to R and write down the matrix M(R).

\item  Prove that if R is a relation and S ⊆ R, then S is a relation.

\item  If R is a reflexive relation on S, then so is any superset of R inside S × S.

\item The following problems pertain to the relationship of congruence mod n, defined on Z as follows:
DEFINITION: Let a and b be integers and let n be a positive integer. Then a ≡n b iff n | (a − b).
Show that 2 | (x−y) iff x and y have the same parity; i.e., either both x and y are even or both are odd.

\item Determine whether the following relations are reflexive, symmetric, or transitive. Prove your claims.
 D = {(x, x) : x ∈ S}, the diagonal of S × S, where S is any set. 
%--------------------------------------------------------------------------
\end{enumerate}




\end{document}
