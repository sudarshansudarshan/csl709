\documentclass[a4paper]{article}

%% Language and font encodings
\usepackage[english]{babel}
\usepackage[utf8x]{inputenc}
\usepackage[T1]{fontenc}
\usepackage{datetime}
\newdate{date}{17}{02}{2017}
\newcommand\tab[1][1cm]{\hspace*{#1}}
\usepackage{amsfonts}

%% Sets page size and margins
\usepackage[a4paper,top=3cm,bottom=2cm,left=3cm,right=3cm,marginparwidth=1.75cm]{geometry}

%% Useful packages
\usepackage{amsmath}
\usepackage{graphicx}
\usepackage[colorinlistoftodos]{todonotes}
\usepackage[colorlinks=true, allcolors=blue]{hyperref}
\usepackage{enumitem}   



\title{Assignment 3}
\author{Deadline}
\date{\displaydate{date}}
\begin{document}
\maketitle

%---------------------------------------------------------------------------


\begin{enumerate}
\item Which of the following statements is/are TRUE for undirected graphs?
P: Number of odd degree vertices is even.
Q: Sum of degrees of all vertices is even. 


\item Let G be a complete undirected graph on 6 vertices. If vertices of G are labeled, then the number of distinct cycles of length 4 in G is equal to........  

\item Simple graphs with V = n, Prove that there are 
\begin{math}
2^ {n \choose 2}
\end {math} such simple graphs exists.

\item Can we have a graph Q with degree sequence (1, 2, 2, 3, 3, 5), loops and parallel edges allowed?

\item A graph G = (V, E) is called bipartite if V can be partitioned into two sets C
and S such that each edge has one vertex in C and one vertex in S. As a specific
example, let C be the set of courses at the university and S the set of students.
Let V = C ∪ S and let {s, c} ∈ E if and only if student s is enrolled in course c.
(a) Prove that G = (V, E) is a simple graph.
(b) Prove that every cycle of G has an even number of edges.

\item What is the maximum number of edges in a bipartite planar graph with n vertices?

\item Can a simple graph have 5 vertices and 12 edges? If so, draw it; if not, explain
why it is not possible to have such a graph. 

\item How many more edges are there in the complete graph $K_7$ than in the complete
graph $K_5$?

\item For a graph G = (V, E), we define the complement $G' =
(V, E')$, where $e \in E'$ if and only if $e \notin E$. Let G be a graph with n
vertices, and let A be its adjacency matrix. Write down the adjacency
matrix A of its complement in terms of A and $M_n$, where $M_n$ is the
adjacency matrix of the complete graph $K_n$.

\item Give an example of a connected graph $G$ where removing any edge of $G$ results in a disconnected graph. Can we determine the number of edges it has?

\item If you construct a graph that captures a relation, describe the properties of a graph representing an equivalence relation.

\item Consider all the possible graphs on $n$ vertices. What can you say about the number of graphs that are connected and those which are disconnected. What fraction of graphs are more in number?

\item We say that G is self-complementary if G is isomorphic to
G. Let $G$ be a cycle on  $n$ vertices. Prove that $G$ is self complementary if $n=5$. Prove that if G is self-complementary, then n ≡ 0 or 1 (mod 4).

\item If there are 15 people in a party, is it possible for each of these people to shake hands with exactly three others?

\item Let $G=(V,E)$ be a loop free undirected graph. Prove that if $G$ contains no cycle of odd length than $G$ is bipartite. 

\item Let $G(V,E)$ be a loop free undirected $n$-regular graph with 
$|V| \leq 2n +2$. Prove that $\overline{G}$ has a Hamilton cycle. 


\item Suppose that someone starts a chain letter. Each person who receives the letter is asked to send
it on to four other people. Some people do this, but others do not send any letters. How many
people have seen the letter, including the first person, if no one receives more than one letter
and if the chain letter ends after there have been 100 people who read it but did not send it out?
How many people sent out the letter?


\item The eccentricity of a vertex in an unrooted tree is the length
of the longest simple path beginning at this vertex. A vertex is
called a center if no vertex in the tree has smaller eccentricity
than this vertex. Show that a tree has either one center or two centers that
are adjacent. 

\item A chessboard has the form of a cross, obtained from a $4 \times 4$ chessboard by deleting the corner squares. Can a knight travel around this
board, pass through each square exactly once, and end on the same square he
starts on?

\item  Can 9 line segments be drawn in the plane, each of which intersects
exactly 3 others?


\end{enumerate}




















\end{document}
